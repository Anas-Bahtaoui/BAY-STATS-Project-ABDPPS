\documentclass[12pt]{article}
\usepackage{amsmath,amsfonts,latexsym,fullpage,graphicx,vmargin,mathrsfs,xcolor}
\usepackage{subcaption}
\usepackage{amsthm}
\usepackage{multirow}
\usepackage{comment}
\usepackage[inline]{enumitem}
\usepackage{hyperref}
\usepackage{natbib}
\usepackage{graphicx}
\usepackage{placeins}

%\urlstyle{same}

\newtheorem{corollary}{Corollary}
\newtheorem{lemma}{Lemma}

% \renewcommand{\baselinestretch}{1.5}

\newcommand{\iid}{\stackrel{\mbox{\scriptsize iid}}{\sim}}
\newcommand{\ind}{\stackrel{\mbox{\scriptsize ind}}{\sim}}
\newcommand{\bm}[1]{\mbox{\boldmath{$#1$}}}
\newcommand{\Xn}{x^n}
\newcommand{\Yn}{y^n}
\newcommand{\ys}{y^*}
\newcommand{\ts}{\theta^*}
\newcommand{\Be}{\rm Beta}
\newcommand{\rpm}{\rm RPM}
\newcommand{\eppf}{\rm EPPF}
\newcommand{\Un}{\rm Unif}
\newcommand{\calB}{\mathcal{B}}
\newcommand{\calN}{\mathcal{N}}
\newcommand{\calP}{\mathcal{P}}
\newcommand{\invWish}{\mathcal{IW}}
\newcommand{\Law}{\mathcal L}
\newcommand{\calU}{\mathcal U}
\newcommand{\ptilde}{\widetilde{p}}
\newcommand{\mutilde}{\widetilde{\mu}}
\newcommand{\dd}{\mathrm d}
\newcommand{\Ga}{\mbox{Gamma}}


%%%%%%%% Mathbb
\newcommand{\Pp}{\mathbb{P}}
\newcommand{\X}{\mathbb{X}}
\newcommand{\Y}{\mathbb{Y}}
\newcommand{\E}{\mathbb{E}}
\newcommand{\R}{\mathbb{R}}


%%%%% COMMENTS
\usepackage{ulem}
\newcommand{\MBnote}[1]{{\color{blue}[\textbf{#1}]}} 
\newcommand{\MBtext}[1]{{\color{blue}{#1}}} 
\newcommand{\MBcanc}[1]{{\color{blue}{\sout{#1}}}} 
\newcommand{\MBsubs}[2]{{\color{blue}{\sout{#1} {#2}}}} 

\newcommand{\JGnote}[1]{{\color{red}[\textbf{#1}]}}
\newcommand{\JGtext}[1]{{\color{red}{#1}}} 
\newcommand{\JGcanc}[1]{{\color{red}{\sout{#1}}}}
\newcommand{\JGsubs}[2]{{\color{red}{\sout{#1} {#2}}}}

% \onehalfspacing

\title{Sparse regression modelling}
\author{Interactive Session \# 2, Bayesian Statistics}
\date{26 Nov. 2021}


\begin{document}
\maketitle

\paragraph{Group Members}

\begin{itemize}
\item Mario Beraha
\item ...
\end{itemize}

\section*{Exercise 1}

\begin{figure}
\includegraphics[width=\linewidth,]{example-image-a}
\caption{Each panel shows the marginal distribution of $\beta_j$ for the six priors under consideration.}
\label{fig:priors}
\end{figure}

\begin{figure}
\includegraphics[width=\linewidth]{example-image-b}
\caption{Unit-balls for the penalty associated to the Normal, Bayesian Lasso and Horseshoe priors. The scale parameter $\tau$ in the different priors is always set equal to $1$.}
\label{fig:balls}
\end{figure}

\paragraph{1.1} The marginal priors for $\beta_j$ are reported in Figure \ref{fig:priors}

\paragraph{1.2} 

\paragraph{1.3} The unit balls are reported in Figure \ref{fig:balls} 

\section*{Exercise 2}

The best predictive performance is obtained with...


\section*{Exercise 3}

The \textbf{names} of the significant variables for the different models are

\begin{itemize}
	\item Normal prior: \\
	P12.A, P16.A ...
	
	\item Bayesian Lasso prior
	
	\item Spike and Slab prior
	
	\item Spike and Slab Lasso prior
	
	\item Horseshoe prior
	
	\item Regularised Horseshoe prior
\end{itemize} 



\section*{}
\bibliographystyle{chicago}
\bibliography{ref_nrmi_factors}




\end{document}
